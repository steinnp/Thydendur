\documentclass[11pt, a4paper]{article}
\usepackage{apacite}
\usepackage{natbib}
\usepackage[utf8x]{inputenc}
\usepackage[T1]{fontenc}
\usepackage[icelandic]{babel}
\usepackage{amsmath, amsthm, amssymb, amsfonts}
\usepackage{graphicx}
\usepackage{tikz}
\usepackage{tkz-euclide}
\usetkzobj{all}
\usepackage{listings}
\usepackage{hyperref}
\usepackage{pgfplots}
\usepackage{geometry}
\usepackage{setspace} 
\usepackage{pdflscape}
\usepackage{mathtools}
\usepackage{fixltx2e}
\usepackage{array}
\setlength\parindent{0pt}
\newtheorem*{problem}{\emph{Problem}}
\newtheorem*{solution}{\emph{Solution}}
\newcommand*\rfrac[2]{{}^{#1}\!/_{#2}} %small fraction
\begin{document}
\begin{titlepage}
\begin{center}

\textsc{\huge Haust 2016}\\[1.5cm]

\textsc{\huge Þýðendur}\\[0.2cm]
\textsc{\huge T-603-THYD}\\[1.5cm]

\end{center}
{ \huge Homework 3\\[1.5cm] }
% Author and supervisor
\Large {
\emph{Nemandi:}\\
Steinn Elliði Pétursson\\[0.5cm]
\emph{Kennitala:}\\
250594-2759\\[0.5cm]
{\large \today}\\[0.5cm]
\emph{Kennari:} \\
Friðjón Guðjohnsen}\\

\end{titlepage}
\leavevmode

\section{}
	Consider the following grammar:\\
		$E~\rightarrow~E~T~O$\\
		$E~\rightarrow~T$\\
		$O~\rightarrow~+$\\
		$O~\rightarrow~-$\\
		$T~\rightarrow~num$\\
	where “+”. “-” and num are tokens. For simplicity all numbers are single digit.\\
	
\subsection*{a)}
	Write a syntax-directed definition (SDD) for the grammar that changes a postfix expression to prefix form. Assume that each nonterminal has the attribute val of type string, and that each attribute value of a node in the parse tree denotes a sub-expression in prefix form.

\begin{solution}\end{solution}~\\
	\begin{tabular}{|l|l|}
	\hline
	\hline
	Production & Semantic rules \\
	\hline
	$E~\rightarrow~E~T~O$ & $E.syn~:=~O.val~||~E_1.syn~||~T.val$\\
	\hline
	$E~\rightarrow~T$ & $E.syn~:=~T.val$\\
	\hline
	$O~\rightarrow~+$ & $O.val~:=~'+'$\\
	\hline
	$O~\rightarrow~-$ & $O.val~:=~'-'$\\
	\hline
	$T~\rightarrow~num$ & $T.val~:=~num.val$\\
	\hline
	\end{tabular}\\
where || is the concatenation operator.

\subsection*{b)}
	Annotate a parse tree for input string 54+3-2+. Note that the attribute value of the root of the parse tree should show the input string in prefix form for the whole expression.

\begin{solution}\end{solution}~\\
\includegraphics[height=12cm]{/Users/steinn/Desktop/sdt.png}

\section{}
	Consider the following grammar for type declarations:\\
	$D~\rightarrow~id~L$\\
	$L~\rightarrow~id~L~|~:T$\\
	$T~\rightarrow~integer~|~real$\\

\subsection*{a)}
	For this grammar, write an SDT (Syntax-Directed Translation Scheme) which sets the type for each name into the symbol table. Use the function addType(X,Y), for which X is a reference/pointer to a symbol table entry and Y is a type.

\begin{tabular}{|l|l|}
\hline
\hline
Production & Actions\\
\hline
$D~\rightarrow~id~L$ & $L~\{addType(id,L.type)\}$\\
\hline
$L~\rightarrow~id~L_1$ & $L_1~\{addType(id,L_1.type)~L.type~=~L_1.type\}$\\
\hline
$L~\rightarrow:T$ & $T~\{L.type~=~T.val\}$\\
\hline
$T~\rightarrow~integer$ & $\{T.val~=~integer\}$\\
\hline
$T~\rightarrow~real$ & $\{T.val~=~real\}$\\
\hline
\end{tabular}


\subsection*{b)}
	Write a recursive-descent parser for the translation scheme you developed. You can assume the function yylex(), which returns the next token from the lexical analyzer. Moreover, assume the function match(Token t), which checks whether the current token matches token t and calls yylex() if that is the case, otherwise it reports an error. In the solution, you should let some functions (that correspond to some non-terminals) return a type. Assume that the type is an integer constant in the form of an enumeration:\\
$enum~TypeCode~\{ INTEGER, REAL, ERROR \}$

\begin{solution}\end{solution}~\\
\begin{lstlisting}[language=Java]
void D() {
	match(tc_id);
	addToken(currentToken, L());
}

int L() {
	if (match(tc_id)) {
		string token = currentToken;
		int type = L();
		addType(token, type);
		return type;
	} else if (match(tc_colon)) {
		return T();
	} else {
		handleError("expected id or colon");
		return 2;
	}
}

int T() {
	if (match(tc_integer)) {
		return 0;
	} else if (match(tc_real)) {
		return 1;
	} else {
		handleError("expected type");
		return 2;
	}
}
\end{lstlisting}

\section{}
	Construct a sequence of TAC instructions (op, arg1, arg2, result) for the statement:\\~\\
	$z = (a+b)∗((c+d)−(−a+b+c))$\\~\\
	Note that * has higher precedence than +. Construct the code in the same way as your own top-down parser would do, i.e. use your changed grammar (the grammar for the Decaf language) to obtain the right precedence order. Moreover, make sure that the order of quadruples mirrors the order that will be generated by your parser.

\begin{solution}\end{solution}~\\
\begin{tabular}{lrrr}
VAR & t1 &&\\
ADD & a & b & t1\\
VAR  & t2 &&\\
ADD   &  c  & d  & t2\\
VAR  &   t3&&\\
UMINUS & a &  t3&\\
VAR  &   t4&&\\
ADD  &   t3 & b &  t4\\
VAR  &   t5&&\\
ADD &    t4 & c  & t5\\
VAR   &  t6&&\\
SUB   &  t2 & t5 & t6\\
VAR  &   t7\\
MULT   & t1&  t6 & t7\\
APARAM & t7&&\\
CALL  &  writeln&&\\
RETURN&&&\\
\end{tabular}
\section{}
	Construct a sequence of TAC instructions (op, arg1, arg2, result) for the code fragment:\\
\begin{lstlisting}
{
	if (i < j) {
		j = j * 2;
	} else {
		j = j / 2;
	} 
}
\end{lstlisting}
\begin{solution}\end{solution}~\\
\begin{tabular}{lrrrr}
&LT&i&j&if\\
&VAR&t1&&\\
&DIV&j&2&t1\\
&ASSIGN&t1&j&\\
&GOTO&ret&&\\
if:&VAR&t2&&\\
&MULT&j&2&t2\\
&ASSIGN&t2&j&\\
ret:&APARAM&j&&\\
&CALL&writeln&&\\
\end{tabular}

\section{}
	Construct a sequence of TAC instructions (op, arg1, arg2, result) for the code fragment:
\begin{lstlisting}
{
int n;
    int sum;
    sum = 0;
    for(n=0;n<10;n++) {
       sum = sum + n * n;
    }
}
\end{lstlisting}

\begin{solution}\end{solution}~\\

\begin{tabular}{lrrrr}
	  &      VAR &    n&&\\
      &  VAR   &  sum&&\\
      &  ASSIGN&  0   &    sum&\\
     &   ASSIGN  &0    &   n&\\
for:   & GE    &  n   &    10   &   ret\\
      &  VAR &    t1&&\\
    &    MULT  &   n  &     n    &   t1\\
      &  VAR    & t2&&\\
    &    ADD  &   sum    & t1   &   t2\\
     &   ASSIGN&  t2    &  sum&\\
     &   VAR &    t3&&\\
    &    ADD  &   1   &    n     &  t3\\
   &     ASSIGN&  t3   &   n&\\
   &     GOTO &   for&&\\
ret: &   APARAM& sum&&\\
   &     CALL& writeln&&\\
        \end{tabular}
        
\section{}
	
What do MSIL (Microsoft Intermediate Language) and Java byte- code have in common? What are their differences?
Use whatever sources you want to find answers to these questions, but make sure that you provide references to your sources in proper scholarly fashion.

\begin{solution}\end{solution}~\\
	
\end{document}